\documentclass[12pt]{article}   	% use "amsart" instead of "article" for AMSLaTeX format
\usepackage{geometry}                		% See geometry.pdf to learn the layout options. There are lots.
\geometry{letterpaper}                   		% ... or a4paper or a5paper or ... 
%\geometry{landscape}                		% Activate for rotated page geometry
%\usepackage[parfill]{parskip}    		% Activate to begin paragraphs with an empty line rather than an indent
\usepackage{graphicx}				% Use pdf, png, jpg, or eps§ with pdflatex; use eps in DVI mode
								% TeX will automatically convert eps --> pdf in pdflatex		
\usepackage{amssymb}

%SetFonts

%SetFonts


\title{Statistics methods in the measurement of radioactivity}
\author{Nikolaas VanKley}
\date{}							% Activate to display a given date or no date

\begin{document}
\maketitle
\begin{abstract}
 Due to the random nature of radioactive decay, it is not possible to make consistent and repeatable measurements of a 
 substance's radioactivity. This stems from the fact that it is impossible to know exactly when a single atom will undergo 
 radioactive decay. However, when looking at a large collection of atoms it is possible to estimate the number of atoms that will
 decay over a certain period of time, which generally serves as a basis for measuring a sample's radioactivity. To achieve this, one must employ statistical methods when analyzing radioactivity data. In this exercise, we will explore the statistical
 methods involved in effectively measuring radioactivity.
\end{abstract}
\pagebreak
\section{Table of Contents}
\section{Introduction}
 The random nature of radioactive decay introduces difficulties into the process of making consistent measurements of the overall
 radioactivity of any given sample, which leads to the question of how to effectively quantify “radioactivity”, or the amount of
 interactions observed due to the energy released during the radioactive decay of atoms. Radioactivity is measured by simply
 counting the number of interactions occurring between the high energy particles (or gamma rays) and a detector over a period of 
 time. Thus, due to the fact that any given atom will decay spontaneously, it is not possible to observe the same number of
 interactions per any given period of time. By employing statistical methods however, it is possible to make inferences about the
 number of interactions that will occur from a large collection of atoms undergoing decay, which turns out to be relatively
 predictable. This forms the basis of measuring radioactivity, and allows one to effectively quantify the overall radioactivity
 of any given sample. In the following experiment, we will be quantifying the radioactivity of a sample emitting gamma radiation.
 First, data on the radiation counts of our sample will be collected using a Geiger-Müller tube, and afterwards the counts will
 be analyzed statistically to obtain an accurate representation of our gamma sample's radioactivity. 
\section{Theory}
\section{Experimental Procedure}
\section{Graphs and Data}
\section{Calculations}
\section{Analysis and Conclusions}
%\section{Bibliography}

%\subsection{}



\end{document}  